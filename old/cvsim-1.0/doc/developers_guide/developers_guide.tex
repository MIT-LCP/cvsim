\documentclass[12pt]{article}
\usepackage[pdftex]{hyperref}



\begin{document}

%%%%%%%%%%%%%%%%%%%%%%%%%%%%%%%%%%%%%
% Title Page
%%%%%%%%%%%%%%%%%%%%%%%%%%%%%%%%%%%%%
\title{CVSim GUI Developer's Guide}
\author{Catherine Dunn}
\maketitle



%%%%%%%%%%%%%%%%%%%%%%%%%%%%%%%
\section{Acknowledgments}
%%%%%%%%%%%%%%%%%%%%%%%%%%%%%%%

This implementation of CVSim is the work of Catherine Dunn with support
by Thomas Heldt and contributions from Brandon Pierquet and Ali Saeed. \\

\noindent Previous implementations were written by Bob Sah (1983), George Moody (1985),
Tim Davis (1989), Rama Mukkamala (2001), Thomas Heldt (2002), and by Eun Bo
Shim and his students (2003). The circulatory model used in CVSim was created
by Roger Mark, based on an analog model of JG Defares and colleagues. The CVSim
model was elaborated by Thomas Heldt, Eun Bo Shim, Roger Kamm, and Roger Mark.
\\

\noindent Development of CVSim was funded by the National Aeronautics and Space
Administration (NASA) through the NASA Cooperative Agreement NCC 9-58
with the National Space Biomedical Research Institute.



%%%%%%%%%%%%%%%%%%%%%%%%%%%%%%%%%%%%%%%%%%%%%%%%%
\section{Running CVSim via Java Web Start}
%%%%%%%%%%%%%%%%%%%%%%%%%%%%%%%%%%%%%%%%%%%%%%%%%

The easiest way to run CVSim is using a web browser with Java Web Start
installed.

\subsection{Supported Platforms}
Running CVSim via Java Web Start has been tested on the following
platforms:

\begin{itemize}
\item Windows XP
\item Mac OS X PPC
\item Mac OS X i386
\item Fedora Core 4, 5, and 7
\item Red Hat Enterprise Linux 4
\end{itemize}

\noindent Running CVSim via Java Web Start is not supported on the following platforms:
\begin{itemize}
\item Solaris
\end{itemize}

\subsection{Installing Java Web Start}
In order to run CVSim from a web browser, you need to
install Java Web Start. Java Web Start is part of the Java Runtime
Engine (JRE). \\

\noindent \textbf{If you are using Windows:} Follow the \href{http://java.sun.com/javase/6/webnotes/install/index.html}{instructions}
on the Sun website to install JRE 6.\\

\noindent \textbf{If you are using Linux:} Follow the \href{http://java.sun.com/javase/6/webnotes/install/index.html}{instructions} on the Sun
website to install JRE 6. You also need to manually 
\href{http://java.sun.com/javase/6/webnotes/install/jre/manual-plugin-install-linux.html}{install} the Java
Plug-in in your web browser.\\

\noindent \textbf{If you are using Mac OS X:} Run Software Update from the System
Preferences. Install Java for Mac OS X 10.4, Release 5.\\

\noindent The first time you 
\href{http://physionet.org/physiotools/cvsim/cvsim_deploy.jnlp}{run CVSim}
, you will see a security warning that
asks you, ``The application's digital signature cannot be verified. Do
you want to run the application''. Click ``Run''. If you want to avoid
that warning in the future, check the ``Always trust content from this
publisher'' box. If you are using Internet Explorer, the security
warning may be displayed behind your browser.



%%%%%%%%%%%%%%%%%%%%%%%%%%%%%%%%%%%%%%%%%%%%%%%%%%%%%%
\section{Building CVSim from the Source Code}
%%%%%%%%%%%%%%%%%%%%%%%%%%%%%%%%%%%%%%%%%%%%%%%%%%%%%%

\subsection{System Requirements}

In order to build CVSim, you need to have the following software installed:

\begin{itemize}
\item gcc, version 4.0.2 or higher 
(See \href{http://gcc.gnu.org}{http://gcc.gnu.org})
\item Simplified Wrapper Interface Generator (SWIG), version 1.3.29 or higher 
(See http://www.swig.org)
\item Java 2 Platform Standard Edition, version 5.0 Update 8 or later
(See http://java.sun.com)
\item GNU Make, version 3.80 or higher 
(See http://www.gnu.org/software/make/)
\end{itemize}

\subsection{Obtaining the source code}

The CVSim source code is available in a gzipped tar archive from
\href{http://physionet.org/physiotools/cvsim/cvsim.tar.gz}{http://physionet.org/physiotools/cvsim/cvsim.tar.gz}.  Download this archive, and then unpack it by:

\begin{verbatim}
     tar xfvz src.tar.gz
\end{verbatim}

This will create a directory named \texttt{cvsim-NN} (where NN is the
release number), containing subdirectories
\texttt{devel} (code) and \texttt{doc} (documentation).

\subsection{Compiling the Code}

\noindent \textbf{If you are using Linux:}
Navigate to the \texttt{devel} directory.  Rename
\texttt{Makefile.include.root.linux} as
\texttt{Makefile.include.root}, and edit this file to 
reflect the location of your installation of gcc, SWIG, and Java. 
At the command line, type \texttt{make}. \\

\noindent \textbf{If you are using Mac OS X:}
Navigate to the \texttt{devel} directory.  Rename
\texttt{Makefile.include.root.osx} as
\texttt{Makefile.include.root}, and edit this file to 
reflect the location of your installation of gcc, SWIG, and Java. 
At the command line, type \texttt{make}. \\

\noindent \textbf{If you are using Windows:} Install and configure cygwin with all
the required packages. Navigate to the \texttt{devel}
directory. Rename \texttt{Makefile.include.root.windows} as
\texttt{Makefile.include.root}, and edit this file to reflect your installation of
gcc, SWIG, and Java. At the command line, type \texttt{make}.



%%%%%%%%%%%%%%%%%%%%%%%%%%%%%%%%%%%%%%%
\section{Running the Code}
%%%%%%%%%%%%%%%%%%%%%%%%%%%%%%%%%%%%%%%

To run the code, navigate to the \texttt{classes} directory and type: \\

\noindent \texttt{java -Djava.library.path=../../lib edu.mit.lcp.CVSim} \\

\noindent The \texttt{-Djava.library.path=../../lib} tells Java where to find
the native C libraries. The CVSim class in the edu.mit.lcp package
contains the \texttt{main()} method.



%%%%%%%%%%%%%%%%%%%%%%%%%%%%%%%%%%%%%%%
\section{The Source Code}
%%%%%%%%%%%%%%%%%%%%%%%%%%%%%%%%%%%%%%%

\subsection{Directory Structure}

What follows is a brief tour of the directory structure. \\ 

\noindent The top level of directories is \\

\noindent \texttt{devel/} \hspace{1cm}
          \texttt{doc/}  \\

\noindent \texttt{devel/} is the main line of development. It's where
all the code is. \texttt{doc/} is for project documentation, including
this guide. In \texttt{devel/}, you will find a couple Makefiles, a shell
script, and the following directories: \\

\noindent \texttt{C/}    \hspace{1cm}     
          \texttt{java/} \hspace{1cm}
          \texttt{lib/}  \hspace{1cm}
	  \texttt{swig/} \hspace{1cm} \\

\noindent \texttt{C/} is for the C code (the model backend),
\texttt{java/} is for the Java code (the GUI frontend), and
\texttt{swig/} is for the SWIG interface that allows the two to work
together. \texttt{lib/} is for the native C libraries generated from
the C code and loaded by Java. The Makefiles are for building CVSim on
Linux and Windows. The shell script, osx\_compile.sh, is for building
the application on Mac OS X. (Using a cross-platform build tool such
as Ant might be better than using a combination of Makefiles and shell
scripts. Alas, I did not have the time to get so fancy.) \\

\noindent Inside the C directory, you will find the source code for the
6-compartment model (\texttt{6\_comp\_model}) and the 21-compartment model
(\texttt{21\_comp\_model}). \\

\noindent The java/ dir is organized in the standard way with separate
directories for source and class files (\texttt{src/} and
\texttt{classes/} respectively) and a subdirectory structure mirroring
the package name (edu/mit/lcp). In other words, the Java source files
(*.java) are in \texttt{src/edu/mit/lcp/} and the class files
(*.class) are in \texttt{classes/edu/mit/lcp}. \\

\end{document}
